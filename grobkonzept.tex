\documentclass[a4paper,titlepage]{scrartcl}
\usepackage{ngerman}
\usepackage{a4wide}
\usepackage{makeidx}
\usepackage{graphicx}
\usepackage{multicol}
\usepackage{float}
\usepackage{color}
\usepackage{textcomp}
\usepackage{alltt}
\usepackage{times}
\usepackage{ifpdf}
\ifpdf
\usepackage[pdftex,
            pagebackref=true,
            colorlinks=true,
            linkcolor=blue,
            unicode
           ]{hyperref}
\else
\usepackage[ps2pdf,
            pagebackref=true,
            colorlinks=true,
            linkcolor=blue,
            unicode
           ]{hyperref}
\usepackage{pspicture}
\fi
\usepackage[utf8]{inputenc}
\makeindex
\setcounter{tocdepth}{3}

\pagestyle{headings}
%\title{2001 A Space Odyssey}
\title{2001 A Space Odyssey - Grobkonzept}
\subject{Semesterarbeit kom2 - Modul 2301b - BFH TI}
\author{Lorenz Schori (schol2@bfh.ch)\\Christoph Isch (ischc2@bfh.ch)}
\publishers{Betreuer: Diego Jannuzzo}

\begin{document}
\maketitle
\newpage

\pagenumbering{arabic}

\section{Grobkonzept}

\subsection{Angaben zum Film}

\begin{description}
    \item[Titel]{2001: A Space Odyssey (2001: Odyssee im Weltraum)}
    \item[Regie]{Stanley Kubrick}
    \item[Erschienen]{1968}
\end{description}

\subsection{Thema Verhandlung}

In Stanley Kubricks Klassiker "`A Space Odyssey"' wird über verschiedene Dinge 
und in unterschiedlicher Konstellation verhandelt. Vielfach geht es dabei darum 
vom Gegenüber Informationen zu erhalten. Während am Anfang des Films vor allem 
unter Menschen (vorher noch unter Affen) verhandelt wird, gesellt sich gegen 
Schluss auch der Supercomputer HAL 9000 dazu.

Folgende Szenen könnten sich für eine eingehende Analyse eignen:

\begin{itemize}
    \item{00:10 Non-Verbal: Affen streiten ums Wasserloch *}
    \item{00:30 Dr. Helmut Floyd - die Russen *}
    \item{00:40 Floyd: Info-Konferenz}
    \item{00:45 Floyd: Expedition zum Quader}
    \item{00:55 BBC Interview mit Crew von Discovery 1 und HAL}
    \item{01:05 HAL - Dave: Persönliche Fragen... *}
    \item{01:15 HAL: Kein Fehler - "`Menschliches Versagen"'}
    \item{01:20 Dave - Frank: Privates Gespräch.}
    \item{01:37 Dave - HAL: "`Öffne das Schleusentor"' *}
\end{itemize}

Szenen, die mit einem Stern (*) bezeichnet sind eignen sich besonders gut.

\subsection{Provisorischer Zeitplan}

\begin{itemize}
    \item{Woche 17: 28.04.2010 Grobkonzept, Dokumente vorbereiten, Sekundärliteratur und Quellen 
        suchen}
    \item{Woche 18: Definitive Auswahl der Szenen, Einleitung verfassen, Hauptteil beginnen}
    \item{Woche 19: Hauptteil abschliessen}
    \item{Woche 20: Zusammenfassung, Schlussfolgerung, Verzeichnisse prüfen und Referenzen
        kompletieren}
    \item{Woche 21: Gegenlesen, 26.05.2010 Abgabetermin}
\end{itemize}
\end{document}
